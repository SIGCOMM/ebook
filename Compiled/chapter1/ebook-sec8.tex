% This work is licensed under
% http://creativecommon.org/licenses/by/3.0/
\section{Design considerations related to mobility}
\label{sec:sec8}

Mobility mechanisms lie close to the heart of networking, so they are
related to many other communication services and aspects of
networking.  In this section, we touch briefly on several other topics
closely related to mobility.

\subsection{Multihoming}
\label{sec:multihoming}

Increasingly, mobile devices connect to the Internet
via multiple interfaces ({\it e.g.,}
a laptop with WiFi and wired Ethernet
interfaces, a smartphone with WiFi and cellular interfaces, or a
virtual machine running on a physical server with multiple wired
Ethernet interfaces).  
Since these interfaces usually connect to
different administrative domains ({\it e.g.,}
a campus WiFi network and a
commercial cellular provider), they must have different IP addresses in
different address blocks.
There is no name for the Internet host itself, and no way to
route to the host itself rather than a specific one of its interfaces.

{\it Sequential multihoming} is the use by a host of multiple
interfaces, one after the other, during the lifetime of a channel.
A mobility mechanism can be used to implement sequential multihoming;
in principle either of the mobility patterns
can be used to provide it.

If we look at the specific real mobility protocols 
in Sections~\ref{sec:sec4} through \ref{sec:sec6}, however,
we see that the DRM protocols in Sections~\ref{sec:sec4} and 
\ref{sec:sec5} work with a single IP address per host, while the
SLM protocols in Section~\ref{sec:sec6} work with multiple IP addresses
per host.
This is an artifact of where these protocols sit in the IP stack rather
than an inherent property of the implementation patterns,
but it does mean that these SLM implementations are currently a better
match for multihoming than these DRM implementations.

{\it Simultaneous multihoming} allows a host to use its multiple
interfaces to contribute bandwidth to the same channel simultaneously.
Simultaneous multihoming
is closely related to sequential multihoming and therefore
to mobility, but not strictly the same because the multiple
interfaces of a host might not be allowed to {\it change} during
the lifetime of a channel.
Of course, what we really want is simultaneous {\it and} sequential
multihoming, which is a
little more complex than sequential multihoming because it requires
protocol extensions so that a layer member can
have and use more than one attachment at a time. 
All of HIP, ILNP, LISP Mobile Node, and Serval
already have session protocols capable of simultaneous multihoming.

Because there is such a close association between host multihoming
and mobility, the well-known multihoming protocols
Multipath TCP~\cite{RFC-6824} 
and the Stream Control Transmission Protocol (SCTP)~\cite{RFC-4960}
are worth studying.
They have been used enough to gather experience on the performance
aspects of switching from one interface to another, and on how performance
affects the buffering and rate control in real transport (session)
protocols such as TCP.
This experience is as relevant to mobility as it is to multihoming.

\subsection{Anycast}

Increasingly, the Internet is a platform for users to
access services hosted on multiple servers in multiple locations.
The appropriate network abstraction for their requirements
is {\it anycast},
in which a service has an {\it anycast name} that corresponds
to a group of servers offering the service.
A request for a communication channel to the service can result in a
channel to any member of the group of servers.

For simple query-response services like DNS,
all server replicas can share a single locator ({\it i.e.,} an IP address),
and rely on IP routing to direct client requests to one of the server
replicas.  
However, IP routing does not guarantee that multiple
messages sent to the same IP address would reach the same server
replica. 
In today's Internet, a domain name
for a geo-replicated service
can map to multiple IP addresses, one for each server replica.
This supports communication services with multiple message exchanges
between client and server.
It is different from mobility, however, because the
higher-level domain name has nothing to do with the channel after
the initial DNS lookup.

Alternatively, anycast could be combined with SLM, as it is in
Serval~\cite{serval,serval-icnp}.
A Serval identifier is an anycast name, and is registered as
located at a dynamic group of servers in the locator layer.
When a request for a channel to an identifier is 
handled by the Serval session protocol,
the protocol selects the locator of some member of the
group.
In contrast to the previous paragraph, the identifier remains
the name of the channel's higher endpoint throughout its lifetime.
The SLM session protocol can thus maintain the channel through both
mobility events and changes to the membership of the server group.

\subsection{Subnetwork mobility}

Mobility proposals typically focus on the
movement of a single mobile endpoint, like a mobile device or a virtual
machine.  However, in some scenarios a subnetwork serving
multiple endpoints can move from one location to another.  
For example, a
fast-moving bus, train, or plane may carry a LAN
that provides network
connectivity to a large collection of passengers.  

Dynamic-routing
mobility in a hierarchical layer
naturally handles subnetwork mobility by updating the routes used
to reach the entire aggregated block of names.
For example, Boeing had an early in-flight WiFi service
that provided seamless mobility for airline passengers by associating
each international flight with an IP address block, and announcing the
block into the global routing system at different locations as the
plane moved~\cite{connexion}.  However, this solution required all
interdomain routers in the Internet to store and update fine-grained
routing information, leading to high overhead.  

In our recent work~\cite{cnm}, we have shown that subnetwork
mobility is merely the mobility of the gateway's attachment to the
larger network, and is implemented with the same two patterns as
mobility of endpoints.
We identified several applications of the design
patterns that seem promising
for handling combinations of subnetwork mobility and endpoint
mobility. 
These solutions have the property that the mechanism for subnetwork
mobility (a bus moves its access point from one roadside LAN to another)
is completely independent of the mechanism for endpoint mobility
(a user with a laptop gets on and off the bus).
Nevertheless, it is not yet clear which solutions for
subnetwork mobility would be most viable in practice.

\subsection{Incremental deployment and interoperation}

Deploying new protocols that span
administrative domains is always challenging, since the Internet
is a federated
infrastructure and cannot easily have everyone upgrade to a new
protocol at the same time.
Most real deployments are DRM mechanisms that
operate within a single administrative domain ({\it e.g.,}
cellular networks, Ethernet LANs, or data-center networks),
or require support only from the mobile endpoint and a small number of
routers ({\it e.g.,} Mobile IPv4).  

It is not surprising that most real mobility implementations use DRM,
because SLM entails many more deployment hurdles.
There can be a new set of identifiers, a new global directory service,
changes to both endpoints, and even
changes to the service interface that are visible to applications.
The early SLM protocol TCP Migrate \cite{tcp-migrate} is probably
the most deployable, requiring only changes to the operating system
at the participating endpoints, but even so it has not had significant
deployment.

Nevertheless, as noted in the introduction,
the pressure for better network mobility support is mounting.
Ubiquitous computing may be a particularly powerful motivator,
because an enormous number of sensors and actuators will require
network access.
This could accelerate the adoption of IPv6, enabling many
other changes in its wake.

For incremental deployment, an SLM-enabled host can interoperate
with a legacy host in a degraded mode in which mobility does not work
but other functions do.
For full mobility, an SLM-enabled host can interoperate with
a legacy host through a proxy or other middlebox.
This raises many new questions concerning how a middlebox is
introduced into the path between the hosts, and on the scalability
of stateful middleboxes.
These new questions must be added to the perennial list of old
interoperability questions, such as how to traverse
NAT boxes.
Identifying
effective ways to deploy these protocols incrementally remains an
active area of research and standards work.

\subsection{Security}

All mobility solutions raise important questions about
security.  
In DRM, who is authorized to announce
routing changes for an address or address block?  
In SLM, who is authorized to update the directory service and a
mobile endpoint's correspondents?  
Answering these questions
successfully requires unforgeable notions of identity, and secure
protocols for sending update messages to routers, directory servers,
and endpoints.  
Can accidental misconfigurations or malicious attacks
overload the routers, directory servers, or endpoints?  Preventing
denial-of-service attacks requires effective ways to limit the work
performed before recognizing that messages are unauthorized.  

As
mentioned in Section~\ref{sec:majordifferences}, some people believe that
SLM protocols face greater security challenges
because arbitrary endpoints can initiate updates to global layer state.
On the other hand, SLM protocols provide for persistent identifiers
which can be used as the basis for authentication of hosts, a valuable
assistance to security.
SLM protocols that use DNS as the directory
service can update DNS
records with an existing secure protocol~\cite{RFC-2137}.  
Some protocols, like HIP and Serval, embed
an endpoint's public key (or a hash of the key) in the identifier; 
these identifiers support secure communication as well as
authentication.
Still, security is a rich and
important topic warranting a much deeper treatment, especially since
new protocols can easily introduce unforeseen vulnerabilities and new
threats.
