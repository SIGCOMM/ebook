% This work is licensed under
% http://creativecommon.org/licenses/by/3.0/
\section{Conclusion}
\label{sec:sec9}

In this chapter we have presented an abstract framework for describing, 
understanding, and comparing approaches to network mobility.  
As illustrations, we have covered several mobility protocols 
in some detail.
We believe the geomorphic model provides a clear and precise way to understand the considerable similarities between different mobility proposals,
allowing discussions to focus on their meaningful distinctions rather than artificial differences in terminology.

We have compared mobility proposals on both qualitative
(deployment constraints, security) and quantitative (resource costs,
latency) criteria.
The basis for making comparisons has been completely {\it structural},
in the sense of {\it structural modeling} as defined in 
Section~\ref{sec:structural}.
This is important because structural comparisons are vastly easier
to obtain than comparisons based on simulation, and should always
be the first step in any evaluation project.

In the interest of brevity, our discussions of quantitative
criteria have merely suggested trends and trade-offs, rather than
providing a more substantive analysis.
A true understanding of metrics such as
storage cost, update cost, and path cost requires a more 
detailed characterization of the proposals, including
supporting technologies such as routing protocols and directory services. 
Scalability depends on how these costs grow as the size of a network
grows within the expected range.
 
Equally important, different mobility mechanisms can be composed.
Even today it would not be surprising to see
dynamic-routing mobility used 
within an administrative domain, while
session-location mobility is used simultaneously
across administrative boundaries.
The ultimate goal would be to compose performance models along with
the mechanisms they are modeling, so that the performance of
a composed solution could be derived from the performance of its
components.   

Today, many Internet applications that could benefit from mobility
use work-arounds instead, satisfying the need for session continuity
with {\it ad hoc} application-specific mechanisms, or simply doing
without \cite{handley}.
This is both an effect and a cause of scant deployment of mobility
mechanisms.
Most existing mobility protocols operate at fairly low levels
in a network architecture, 
specifically the link, network, and transport levels
of the classic Internet stack.
At these levels they are expensive, difficult to deploy, or both.

Many of these limitations are unnecessary, as
the essence of mobility is simply a dynamic binding of more abstract
names to more concrete names.  
As such, mobility can be easily implemented in middleware as a service to
even higher-level application layers.  
We believe that this is a fruitful avenue for further exploration,
particularly because it might be easier to optimize narrowly-targeted
implementations of mobility.

More generally, we believe that the geomorphic view promotes
common terminology, modularity,
separation of concerns, discovery of design patterns,
composition,
rigorous reasoning,
and code reuse in networking.
While widely appreciated by software engineers, 
these concepts have been less central to the study of networking.  
We believe that the geomorphic view should be extended to understand
other important aspects of networking.
We also believe that an appreciation of these concepts would be 
valuable for networking researchers and practitioners alike, 
far beyond the treatment of any one subject like network mobility.
