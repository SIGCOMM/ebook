
 
In the chapter on Internet Topology Research Redux 
we revisit some of the interesting properties of the critical
infrastructure at router, AS and PoP level which today underpin how
the world is knot together. 

On top of this ever evolving topology,
traffic sources are regulated by end-to-end protocols such as TCP -
these too see continual changes, and the chapter on
Recent Advances in Transport Protocols covers some of the ways TCP is
being enhanced to cope with multihoming, and take advantage of multiple
active routes between ends, whether in the data center or in the
pocket.

We can put together the traffic sources, and the topology
under one framework, that of Internet Traffic Matrices. The next
chapter introduces a primer to this topic.

One general framework for understanding dynamic traffic management in
the Internet is to think of the system as one of continual
optimization. The chapter on Optimizing and Modeling Dynamics in Networks
revisits the goals of fairness and the control problem and its
solution space.

It has been said that all is fair in love and war, but the only
certainties in the world are death and taxes. Life may not be fair,
but we can enforce a different kind of fairness on the Internet
through traffic pricing. The next chapter discusses
Smart Data Pricing (SDP) and covers a range of 
Economic Solutions to Network Congestion, bringing game theory to bear
on the problem where the previous chapter engaged with the weapons of
optimization.

At a coarser grain (in time and space) we can control traffic by
partitioning our network into VPNs - the next chapter looks at the
practical tools available to the operator to manage a set of such
disjoint systems and their capacity, by focusing on MPLS and Virtual
Private Networks.

For some time, the capacity of the net has been consumed largely by
people downloading or live streaming content. Content Distribution
Networks are overlays that allow management of the load. A less
rigorous but nearly as popular family of tools exist based on the
famous peer-to-peer paradigm. The next chapter looks at how
the world has moved on from fighting, to embracing
Collaboration Opportunities for Content Providers and Network
Infrastructures.

The true explosion in Internet access in the developing world (and now
dominant form of access in the developed world too) has been through
mobile devices (smart phones, tablets and the rest). The next chapter
looks at the Design Space of Network Mobility, and how seamlessness is
achieved (or at least as close to seamless as we can get today).

Finally, there have been sporadic attempts to build provider-less
networks that are build out of mobile wireless nodes only. It is quite
a challenge, and a number of breakthroughs over the last few years have
led to engineering solutions for Enabling Multihop Communication in
Spontaneous Wireless Networks, which are starting to look like a
viable alternative to managed wireless networks in some niche areas.

This is an exciting time to be learning about communications and
building and extending systems for communications. Planet Earth is not
far away from being 100\% connected, and the capacity and functionality
that has been achieved over the last few decades is quite astounding.
It does not look like we have hit any fundamental limits, nor
will for some time to come!

Read this book by some of the world's leading lights in the area 
of communications and enjoy.

