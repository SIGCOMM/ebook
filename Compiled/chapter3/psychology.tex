\section{Psychological Aspects of SDP}\label{sec:psychology}
 
Arguably, the foremost factor in realizing a successful data plan is its adoptability by end users. But this depends not only on good economic models and system capabilities, but also on understanding consumer behavior and psychological aspects, and in particular, how client-side interfaces need to be designed to make them effective. These concerns require the networking researchers to interact and work closely with the ongoing efforts in the human-computer interaction (HCI) community. This section covers some of the basic themes emerging from the HCI research on psychology of Internet users.

As early as 2005, HCI researchers introduced networking, particularly user behavior considerations in home networks, as an important aspect of HCI studies \cite{Grinter2005}. In the context of broadband networks, few HCI works have developed human-facing systems to manage network usage.  The \emph{Eden} system \cite{Eden} modified a home router to provide users with an intuitive interface for managing their ``home network experience." The main focus of such home-networking tools has been on designing GUIs to help users understand the physical location of different devices in their home and be able to perform basic administrative functionalities like perform membership management, access control, network monitoring, etc. In a similar vein, the \emph{Homework project} \cite{Mortier2012} modifies the handling of protocols and services at the home router to monitor data usage, prioritize different devices, and monitor other users� data consumption (usually in the context of parental control) in order to reflect the interactive needs of the home. Other studies have focused on understanding the impact of monitoring and sharing bandwidth speed in a wired home network \cite{chetty2011my}. The authors carried out a field trial with 10 households using a visual network probe designed to help educate and empower consumers by making broadband speeds and sources of slow-downs more visible. In a separate work, Chetty et al. \cite{chetty2010s} also carried out a field trial of their \emph{Home Watcher} bandwidth management tool to study the effect of viewing others� bandwidth usage on social dynamics in the household. They showed that when resource contention amongst different household members is made visible, users have better understanding of bandwidth their usage and allocation, revealing new dimensions in household politics. Yet while these works addressed several crucial elements of network intervention (e.g., throttling, capping, parental control) and its related visualization tools, they have been limited to either modifying the network stack within the OS \cite{chetty2010s} or deploying a custom-built access point \cite{Mortier2012}. In contrast, the need today is to focus on understanding the following issues: (a) how economic incentives (i.e., pricing) can impact and modify mobile user behavior, (b) how researchers can carry out field trials (almost seamlessly from an end-user perspective) by interposing between the ISP and its real customers (i.e., participants). 

In the context of mobile networks, \cite{Roto2006} considers the need for mechanisms to help users track their spending on mobile data. This line of research on user interface designs needs more attention to realize more dynamic pricing plans and study user response to such plans. However, realizing such data plans requires assumptions of rational behavior to be realized in practice, i.e., that people perceive the pricing signals and change their behavior. The Berkeley INDEX project \cite{INDEX} in the 1990s for wired Internet suggested that users should be able to view prices and select desired QoS levels (i.e., bandwidth speed), with similar results reported by the M3I project \cite{M3I}. 
Similar field trials on the HCI aspects of time-dependent pricing for mobile data have been reported in \cite{sigchi}, and the main themes emerging from these studies will be discussed below.

Recently, there have been calls to investigate HCI aspects of time-dependent pricing in the context of ecological sustainability and energy consumption \cite{Pierce2012}. Even without economic incentives, good UI designs of energy monitors have been shown to be effective in changing consumer behavior \cite{Froelich2010,Pierce2012}. These investigations have ranged from a large-scale media art installation visualizing energy consumption in an office building, to power strips that change color to show the energy used by individual electrical sockets \cite{Heller2011,Holmes2007}. One of the usual tradeoffs in visualizing energy usage is the use of pictorial versus numerical usage amounts. For instance, the iPhone application \emph{WattBot} allows users to monitor their home energy usage, with colors indicating usage amounts \cite{Petersen2009}. While the colors enabled users to quickly grasp their qualitative energy usage, users also wanted to track their evolving usage behavior by viewing their usage history \cite{Chetty2008}. In addition to the ``manner" of presenting usage data, users expressed concern over the ``convenience" of checking their usage. For instance, researchers testing a desktop widget that showed computer energy efficiency found that users appreciated the inconspicuous, easy-access nature of the widget \cite{Kim2010}. Many of these design insights were incorporated in the client-side UI design of the TDP trial for mobile users reported in \cite{sigchi}. Another key consideration in designing such mobile applications for smarter data plans is that the GUIs need to be account for the form-factor, presentation, and convenience of use on mobile devices and platforms. Incorporating features like parental control and usage history have been found to be desirable to users, even in the TDP paradigm \cite{sigchi,datawiz}.  

The main themes emerging from these research on HCI aspects of networking are that:
\begin{enumerate}
\item Consumers are very concerned about the increasing cost of data plans but are not fully aware of their monthly usage. Therefore mechanisms that help them to better monitor and control their own usage and allocate bandwidth among household members (or in shared data plans for mobile) are perceived to be useful by users. Moreover, given the right economic incentives, many are willing to wait for some form of a feedback signal from the network on the congestion levels and change their behavior, e.g., choose higher QoS service for critical applications in smart market scenarios \cite{M3I}, or wait for discounted periods for non-critical applications in case of time-dependent data plans \cite{sigchi}. This is therefore a promising direction in the evolution of Internet pricing, and follows similar trends in other markets like electricity and transport networks. 
\item The key factor in enabling smart data plans is designing the user interfaces that are effective and understandable to users. Previous research has shown that well-designed user-side applications can not only help users make decisions on deferring usage, but also can become a tool that helps them to self-educate, monitor, and control their usage and spending. Among the features found to be effective as means of communicating useful information to users were color-coding\footnote{While color-coding has been used and appeared effective in several scenarios, care must be taken to provide a secondary signal for persons suffering from color-blindness.} of price/discounts, information indicators (e.g., current price) on the home screen of devices, usage history, recommendations on usage deferrals, etc.
\item Past research also revealed interesting insights on the user psychology of controlling their usage. First, trial participants viewed parental control at the granularity of apps as being useful in managing their usage. Second, they showed a desire to control their usage manually instead of delegating control to an autopilot mode. When coupled with the desire for parental control, we find users want to take charge of their consumption behaviors, for themselves and their families, in ways that require transparency and flexibility. This tradeoff between users' desire for transparency and control will be an important design considerations for enabling smarter data plans.
\item Consumers prefer a certain degree of certainty in the prices, which explains the popularity of flat-rate plans. Therefore for dynamic and smarter data plans to succeed, the time granularity of changes in prices have to be carefully accounted for. Users want to know and have certainty about these prices/discounts in advance, and therefore naive dynamic pricing plans may not work. In particular, real-time pricing that provides price signals based on current congestion levels may demand a higher degree of user engagement (or automation) than users are willing to adopt. Plans that have been explored in energy markets, such as ``day-ahead'' pricing \cite{ha2012tube,sigchi}, therefore may be more acceptable to users.
\item As both networking and HCI address increasingly complex socio-technical ecosystems, e.g., mobile Internet, cloud computing, smart grids, etc., incorporating economic analysis as a part of user behavior studies is important. There is also a need to understand how researchers can create new frameworks for realistic experiments and field trials on the HCI aspects of network economics. Various research ideas have taken different systems approaches, from modifying the network stack within the OS \cite{chetty2010s} to deploying custom-built access points \cite{Mortier2012}, to the TUBE project \cite{ha2012tube} which developed a system for researchers to act as a resale ISP by interposing between ISPs and their customers, and provided these two sides with client-side mobile applications and ISP-side incentive computation modules.  
\end{enumerate}
 
  
% \cite{Roto2006} 	Roto, V., Geisler, R., Kaikkonen, A., Popescu, A. and Vartiainen, E. Data traffic costs and mobile browsing user experience. In Proc. Workshop on Empowering the Mobile Web, in conjunction with WWW 2006.
%\cite{Grinter2005}	Grinter, R.E., Edwards, K. W. and Newman, M. W. The work to make a home network work. In Proc. ECSCW, Springer (2005): pp. 469 � 488.
%\cite{Eden}	Yang, J., Edwards, W. K. and Haslem, D. Eden: Supporting home network management through interactive visual tools. In Proc. UIST, ACM Press (2010): pp. 109 � 118.  
%\cite{Mortier2012} Mortier, R., Rodden, T., Tolmie, P., Lodge, T., Spencer, R., Crabtree, A., Koliousis, A. and Sventek, J. Homework: Putting interaction into the infrastructure. In Proc. UIST, ACM Press (2012): pp. 197� 206.
%\cite{Chetty2011} Chetty, M., Haslem, D., Baird, A., Ofoha, U., Sumner, B. and Grinter, R. Why is my Internet slow?: Making network speeds visible. In Proc. CHI 2011, ACM Press (2011): 1889 � 1898.
%\cite{Chetty2010} Chetty, M., Banks, R., Harper, R., Regan, T., Sellen, A., Gkantsidis, C., Karagiannis, T. and Key, P. Who�s hogging the bandwidth?: The consequences of revealing the invisible in the home. In Proc. CHI 2010, ACM Press (2010): pp. 659 � 668.
%\cite{Pierce2012} Pierce, J. and Paulos, E. Beyond energy monitors: Interaction, energy, and emerging energy systems. In Proc. CHI 2012, ACM Press (2012): 665 � 674.
%\cite{Froelich2010} Froelich, J., Findlater, L. and Landay, J. The design of eco-feedback technology. In Proc. CHI 2010, ACM Press (2010): 1999 � 2008.
%\cite{Heller2011} Heller, F. and Borchers, J. PowerSocket: Towards on-outlet power consumption visualization. In Ext. Abstracts CHI 2011, ACM Press (2011): 1981 � 1986.
%\cite{Holmes2007}Holmes, T. Eco-visualization: Combining art and technology to reduce energy consumption. In Proc. C&C 2007, ACM Press (2007): 153 � 162.
%\cite{Petersen2009} Petersen, D., Steele, J. and Wilkinson, J. WattBot: A residential electricity monitoring and feedback system. In Ext. Abstracts CHI, ACM Press (2009): 2847 � 2852.
%\cite{Chetty2008} Chetty, M., Tran, D. and Grinter, R. Getting to green: Understanding resource consumption in the home. In Proc. UbiComp 2008, ACM Press (2008): 242 � 251.
%\cite{Kim2010} Kim, T., Hong, H. and Magerko, B. Design requirements for ambient display that support sustainable lifestyle. In Proc. DIS 2010, ACM Press (2010): 103 � 112.
