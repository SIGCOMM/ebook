\section{Engineering Aspects of SDP}\label{sec:engg}

Following the works of congestion pricing by Kelly, Gibbens, and others \cite{Kelly-eff,gibbens-kelly,kelly1998rate,kelly2000}, researchers have focused on developing the basic idea of pricing bandwidth resources according to congestion marks (i.e., price ``pollution" instead of just volume). Congestion marks on packets can be used as a mechanism to indicate to the end-users or end-systems (or ``agents") that they may need to pay more to avail a certain level of QoS when the network gets congested. A team of researchers, led by Briscoe,\footnote{http://bobbriscoe.net/pubs.html} today are developing these ideas on the accounting architecture, feedback aspects of congestion signaling, incentive compliance, security etc. in conjunction with working groups like WG CONEX in the IETF\footnote{http://datatracker.ietf.org/wg/conex/charter/} that focus on implementation aspects. One such EU collaborative project, the \emph{M3I} (Market Managed Multi-service Internet) has proposed an architecture for market management of the Internet's QoS and demonstrated new business models that it can enable. The core of the QoS problem tackled by the M3I project is to solve the fast control problem to avoid QoS degradation during short-term congestion. In doing so, the M3I architecture requires network providers to deploy ECN (Explicit Congestion Notification) on all routers so that the congestion experienced field in the IP packet header can be set with a probability related to the current network load, allowing prices to adapt to the congestion conditions in the network. The M3I solution can be realized in different ways and in different scenarios, which have been extensively discussed in \cite{M3I}. One solution uses these pricing feedback signals to encourage self-admission control. In effect this is real time pricing (RTP) version of time-of-day pricing. A second solution synthesizes admission control at the network edge from a dynamically priced wholesale service.\footnote{http://bobbriscoe.net/projects/m3i/dissem/cc02/m31\_cc02.pdf} The M3I technology has also implemented several business models over different QoS mechanisms as discussed in \cite{M3I-business}.  In a similar vein, \cite{Chiu2008} has used game-theoretic model to explore the uplink pricing as a way to provide differential pricing to P2P and regular users in a competitive market.

Like M3I, much SDP research realizes the vision of pushing control out of the network into the hands of the end-users. However, giving users more usage control has to be introduced carefully so that the users are not overwhelmed with choices and network providers do not lose revenue. Users typically prefer to have some certainty about their monthly bills and hence as opposed to real-time dynamic pricing, some SDP researchers have looked at day-ahead time-dependent pricing. But day-ahead pricing requires network providers to be able to optimize for the future prices they are willing to offer based on past usage patterns and their prediction of user elasticity of demand for various types of traffic. On the user-side, it requires developing interfaces through which users can provide their feedback to these prices. 

The key subsystems required in realizing the basic feedback-control architecture envisioned in M3I \cite{M3I} and SDP are:
\begin{enumerate}
\item \emph{Tariff Communication} is used to distribute pricing policy to other subsystems. The Price Communication Protocol (aka Tariff Distribution Protocol \cite{protocol}) is a flexible protocol that can use different transport mechanisms like UDP multicast, HTTP and RSVP to distribute tariffs between the ISP's management systems and to customers. 
\item Charging, Authentication, and Accounting module applies the chosen tariff plan to measured usage data for billing purposes.
\item Price Computation calculates optimal prices to offer to the user. In M3I, this price adapts to the real time network load while in SDP the prices are computed in advance to provide guarantees to the user's expenses. 
\item Optimization and Prediction Modules are used by the provider to predict the future congestion levels and estimate the users' patience across different traffic classes at different types of traffic based on the history of traffic volumes and the observed changes to it in reaction to offered prices. The optimization module then computes the optimal prices (or discounts) to offer to the end-users in the SDP architecture.
\item Data Gathering/Usage Metering is used by the provider as inputs to the charging and accounting systems and for price calculation.
\item Mediation is used to aggregate gathered data and do format conversions as needed.
\item Client Scheduler is used in the SDP framework to create an ``autopilot" mode of operation for the end-user which schedules the various types of application traffic based on user's specified delay elasticities and offered future prices so as to minimize user expenses.
\item Charge Reaction allows customers to react to the offered prices which gets fed back into the feedback-control loop between the ISP and its customers. In some scenarios such function is performed by a software agent on the end-user device (e.g., ``autopilot" mode in SDP or the Dynamic Price Handler in M3I), while in others, as in day-ahead TDP, user interfaces need to be provided for users to react to the offered prices.  
\end{enumerate}       

Next we discuss a case study of a particular realization of SDP --dynamic day-ahead time-dependent pricing; the model considerations for it, system architecture, HCI design aspects and results from a field trail.  