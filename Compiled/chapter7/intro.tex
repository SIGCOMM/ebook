\section{Introduction}

Internet topology research is concerned with the study of the various
types of connectivity structures that are enabled by the layered
architecture of the Internet. More than a decade of Internet topology
research has produced a number of high-profile "discoveries" that
continue to fascinate the scientific community, even though (or,
especially because) they have been simultaneously touted by different
segments of that community as either seminal, controversial, seriously
flawed, or simply wrong.  Among these highly-popularized discoveries
are the observed power-law relationships of the Internet topology, the
network's scale-free nature, and its extreme vulnerability to attacks
that target the highly-connected nodes in its core (\ie the
Achilles' heel of the Internet).

The purpose of this chapter is to bring order to the current state of
Internet topology research and separate ``the wheat from the chaff''.
In particular, by relying on carefully vetted data and readily
available domain knowledge, we re-examine the reported discoveries and
expose them to higher standards with respect to statistical inference
and model validation.  In the process, we reveal the superficial
nature of many of these discoveries and provide alternative solutions
that reflect networking reality and do not collapse under scrutiny
with high-quality data or when examined in detail by domain experts.

\subsection{The many facets of Internet connectivity}

Internet topology research is concerned with the study of the various
types of connectivity structures that are enabled by the layered
architecture of the Internet.  These structures include the inherently
physical components of the Internet's infrastructure (\eg routers
and switches and the fiber cables connecting them) as well as a wealth
of more logical topologies that can be defined and studied at the
higher layers of the Internet's TCP/IP protocol stack (\eg IP-level
graph, AS-level network, Web-graph, P2P networks, Online Social
Networks or OSNs).


As early as the ARPANET\index{ARPANET}, researchers were drawing maps
of the network representing its
connectivity~\cite{cerf90:_selected_maps}.  The earliest date to
1969. In those days, the entire network was simply enough to draw on
the back of an envelope\footnote{For instance see
  \url{http://personalpages.manchester.ac.uk/staff/m.dodge/cybergeography/atlas/roberts_arpanet_large.gif}},
and accurate maps could be drawn because every piece of equipment was
expensive, installation was a major task, and only a few people worked
on the network.

As the network grew, its complexity also grew, until the point where
no one person could draw such a map. At that point, automated
strategies started to arise for measuring the topology. The earliest
Internet topology studies date back to the time of the
NSFNET\index{NSFNET} and focused mainly on the network's physical
infrastructure consisting of routers, switches and the physical links
connecting them (\eg
see~\cite{Calvert97,pansiot98:_inter}). The decommissioning of
the NSFNET around 1995 led to a transition of the Internet from a
largely monolithic network structure (\ie NSFNET) to a genuinely
diverse "network of networks."  Also known as Autonomous Systems
(ASes), together these individual networks form what we now call the
"public Internet" and are owned by a diverse set of organizations
and companies that includes large and small Internet Service Providers
(ISPs), transit providers, network service providers, Fortune 500
companies and small businesses, academic and research organizations,
content providers, Content Distribution Networks (CDNs), Web hosting
companies, and cloud providers.

With this transition came an increasing fascination of the research
community with a largely economics-driven connectivity structure
commonly referred to as the Internet's AS-graph; that is, the logical
Internet topology where nodes represent individual ASes and edges
reflect observed relationships among the ASes (\eg
customer-provider, peer-peer, or sibling-sibling relationship). It is
important to note that the AS-graph\index{AS-graph} says little about
how two ASes connect with one another at the physical level; in
particular, it says nothing about if or how they exchange actual
traffic.  Nevertheless, starting shortly after 1995, this fascination
with the AS-graph has resulted in thousands of research publications
covering a range of aspects related to measuring, modeling, and
analyzing the AS-level topology of the Internet and its evolution over
time~\cite{govindan97:_topology,roughan11}.

At the application layer, the emergence of the World Wide Web (WWW) in
the late 1990 as a killer application generated general interest in
exploring the Web-graph, where nodes represent web pages and edges
denote hyperlinks~\cite{Broder:2000:GSW:346241.346290}. While this overlay network
or logical connectivity structure says nothing about how the servers
hosting the web pages are connected at the physical or AS level, its
scale and dynamics differ drastically from its physical-based or
economics-driven underlays -- a typical Web-graph has billions of
nodes and even more edges and is highly dynamic; a large ISP's
router-level topology consists of some thousands of routers, and
today's AS-level Internet is made up of some 30,000-40,000 actively
routed ASes and an order of magnitude more links.

Other applications that give rise to their own "overlay" or logical
connectivity structure and have attracted some attention among
researchers include email and various P2P systems such as Gnutella,
Kad, eDonkey, and BitTorrent\index{P2P}\index{BitTorrent}.  More
recently, the enormous popularity of Online Social Networks (OSNs) has
resulted in a staggering number of research papers dealing with all
different aspects of measuring, modeling, analyzing, and designing
OSNs\index{OSN}.  Data from large-scale crawls or, in rare
circumstances, OSN-provided data have been used to examine snapshots
of many real-world OSNs or OSN-type systems, where the snapshots are
generally simple graphs with nodes representing individual users and
edges denoting some implicit or explicit friendship relationship among
the users.

\subsection{Many interested parties with different objectives}

The above-mentioned list of possible connectivity structures that
exist in today's Internet is by no means complete, but illustrates how
these structures arise naturally within the Internet's layered
architecture. It also highlights the many different meanings of the
term ``Internet topology,'' and sensible use of this term requires
explicitly specifying which facet of Internet connectivity is
considered because the differences are critical.

The list also reflects the different motivations that different
researchers have for studying Internet-related graphs or networks.
For example, engineers are mainly concerned with the physical facets
of Internet connectivity, where technological issues generally
dominate over economic and social aspects.  However, the more
economics-minded researchers are particularly interested in the
Internet's AS-level structure where business considerations and market
forces mix with technological innovation and societal considerations
and shape the very structure and evolution of this logical topology.
Moreover, social scientists see in the application-level connectivity
structures that result from large-scale crawls of the various OSNs new
and exciting opportunities for studying different aspects of human
behavior and technology-enabled inter-personal communication at
previously unheard of scale.

In addition, mathematicians are interested in the different
connectivity structures mainly because of their many novel features
and properties that tend to require new and creative modeling and
analysis methodologies. From the perspective of many computer
scientists, the challenges posed by many of these intricate
connectivity structures are algorithmic in nature and arise from
trying to solve specific problems involving a particular topological
structure. For yet another motivation, many physicists turned
network scientists see the Internet as one of many examples of
large-scale complex networks that awaits the discovery of universal
properties that do not depend on system-specific details and advance
our understanding of these complex networks irrespective of the domain
in which they arose in the first place.

\subsection{More than a decade of Internet topology research}

When trying to assess the large body of literature in the area of
Internet topology research that has accumulated since about 1995 and
has experienced enormous growth especially during the last 10+ years,
the picture that emerges is at best murky.

On the one hand, there are high-volume datasets of detailed network
measurements that have been collected by domain experts.  These
datasets have been made publicly available so other researchers can
use them.  As a result, Internet topology research has become a prime
example of a measurement-driven research effort, where third-party
studies of the available datasets abound and have contributed to a
general excitement about the topic area, mainly because many of the
inferred connectivity structures have been reported to exhibit
surprising properties (\eg power-law relationships for inferred
quantities such as node
degree~\cite{faloutsos99:_power_law_relat_of_inter_topol}). In turn,
these surprising discoveries have led network scientists and
mathematicians alike to develop new network models that are provably
consistent with some of this highly-publicized empirical evidence.
Partly due to their simplicity and partly due to their strong
predictive power, these newly proposed network models have become very
popular within the larger scientific
community~\cite{barabasi00,barabasi09:_scale_free_networ,barabasi12}. For
example, they have resulted in claims about the Internet that have made
their way into standard textbooks on complex networks, where they are
also used to support the view that a bottom-up approach dominated by
domain-specific details and knowledge is largely doomed when trying to
match the insight and understanding that a top-down approach centered
around a general quest for "universality" promises to
provide~\cite{barabasi02:_linked,dorogovtsev03:_evolut_networ,pastor-satorras04:_evolut_struc_inter,newman10:_networ}.

On the other hand, there is a body of work within the networking
research literature that argues essentially just the opposite and
presents the necessary evidence in support of a inherently
engineering-oriented approach filled with domain-specific details and
knowledge~\cite{Li04,Alderson05,willinger09:_mathem_and_inter}. In contrast to being
measurement-driven, this approach is first and foremost concerned with
notions such as a network's purpose or functionality, the hard
technological constraints that the different devices used to build a
network's physical infrastructure have to obey, or the sources of
uncertainty in a network's "environment" with respect to which the
built network should be robust.  As for the measurements that have
been key to the top-down approach, the reliance on domain knowledge
reveals the data's sub-par quality and highlights how errors of
various forms occur and can add up to produce results and claims that
create excitement among non-experts but quickly collapse when
scrutinized or examined by domain experts. While there exist currently
no textbooks that document these failures of applying detail- and
domain knowledge-agnostic perspective to the Internet, there is an
increasing number of papers in the published networking research
literature that detail the various mis-steps and show why findings and
claims that look at first glance impressive and conclusive to a
science-minded reader turn out to be simply wrong or completely
meaningless when examined closely by domain
experts~\cite{willinger09:_mathem_and_inter,Alderson10,krishnamurthy11:_what}.

In short, a survey of the existing literature on Internet topology
research leaves one with the distinct impression that ``too many cooks
spoil the broth.''  We hope that in the not-too-distant future, this
impression will be replaced by ``many hand make light work'', and we
see this chapter as a first step towards achieving this goal.

\subsection{Themes}

In writing this chapter there are a number of themes that emerge, and
it is our intention to highlight them to bring out in the open the
main differences between a detail-oriented engineering approach to
Internet topology modeling versus an approach that has become a
hallmark of network science and aims at abstracting away as many
details as possible to uncover ``universal'' laws that govern the
behavior of large-scale complex networks irrespective of the domains
that specify those networks in the first place.
 
\vspace{5mm}

\begin{description}
\item[{\bf Theme 1:}] When studying highly-engineered systems such as
  the Internet, ``details'' in the form of protocols, architecture,
  functionality, and purpose matter.

\item[{\bf Theme 2:}] When analyzing Internet measurements, examining
  the ``hygiene'' of the available measurements (\ie an in-depth
  recounting of the potential pitfalls associated with producing the
  measurements in question) is critical.

\item[{\bf Theme 3:}] When validating proposed topology models, it is
  necessary to treat network modeling as an exercise in
  reverse-engineering and not as an exercise in model-fitting.

\item[{\bf Theme 4:}] When modeling highly-engineered systems such as
  the Internet, beware of M.L. Mencken's quote ``For every complex
  problem there is an answer that is clear, simple, and wrong.''

\end{description}
