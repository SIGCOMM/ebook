%%%%%%%%%%%%%%%%%%%%%%%%%%%%%%%%%%%%%%%%%%%%%%%%%%%%%%%%%%%%%%%%%%
\section{Incentives for Collaboration}\label{sec:Incentives}
%%%%%%%%%%%%%%%%%%%%%%%%%%%%%%%%%%%%%%%%%%%%%%%%%%%%%%%%%%%%%%%%%%

ISPs are in a unique position to help CDIs and P2P systems to improve content
delivery.  Specifically, ISPs have the knowledge about the state of the
underlying network topology and the status of individual links that CDIs are
lacking. This information not only helps CDIs in their user-to-server mapping,
but also reduces the need for CDIs to perform large-scale active measurements
and topology discovery \cite{sureroute}.  It also enables CDIs to better
amortize their existing infrastructure, offer better quality of experience to
their users, and plan their infrastructure expansion more efficiently. On the
other side, ISPs are not just selflessly giving up their network information.
Offering their intimate knowledge of the network to CDIs puts ISPs in the
position that they can also actively guide the CDIs. This allows ISPs to gain
unprecedented influence on CDI traffic.

The opportunity for ISPs to coordinate with CDIs is technically possible
thanks to the decoupling of server selection from content delivery. In general,
any end-user requesting content from a CDI first does a mapping request,
usually through the Domain Name System (DNS). During this request, the CDI
needs to locate the network position of the end-user and assign a server
capable of delivering the content, preferably close to the end-user.  ISPs have
this information ready at their fingertips, but are currently not able to
communicate their knowledge to CDIs. Furthermore, ISPs solve the challenge of
predicting CDI traffic, which is very difficult due to the lack of information
on the CDI mapping strategy regarding the end-users to servers assignment. In
order to reap the benefits of the other's knowledge, both parties require
incentives to work together.

\subsection{Incentives for CDIs}

The CDIs' market requires them to enable new applications while reducing
their operational costs and improve end-user experience~\cite{Akamai-Network}.
By cooperating with an ISP, a CDI improves the mapping of end-users to servers,
improves in the end-user experience, has accurate and up-to-date knowledge of
the networks and thus gains a competitive advantage. This is particularly
important for CDIs in light of the commoditization of the content delivery
market and the selection offered to end-users, for example through
meta-CDNs~\cite{Conviva2011}. The improved mapping also yields better
infrastructure amortization and, thanks to cooperation with ISPs, CDIs will no
longer have to perform and analyze voluminous measurements in order to infer
the network conditions or end-user locations.

To stimulate cooperation, ISPs can operate and provide their network
knowledge as a free service to CDIs or even offer discounts on peering or
hosting prices, \eg for early adopters and CDIs willing to cooperate. The loss
of peering or hosting revenue is amortized with the benefits of a lower network
utilization, reduced investments in network capacity expansion and by taking
back some control over traffic within the network. Ma et al.~\cite
{CooperativeSettlement:ToN} have developed a methodology to estimate the prices
in such a cooperative scheme by utilizing the Shapley settlement mechanism.
Cooperation can also act as an enabler for CDIs and ISPs to jointly launch new
applications in a cost-effective way, for example traffic-intensive
applications such as the delivery of high definition video on-demand, or
real-time applications such as online games.

\subsection{Incentives for ISPs}\label{sec:ISP-Incentive} 

ISPs are interested in reducing their operational and infrastructure upgrade
costs, offering broadband services at competitive prices, and delivering the
best end-user experience possible. Due to network congestion during peak hour,
ISPs in North America have recently revisited the flat pricing model and some
have announced data caps to broadband services. A better management of traffic
in their networks allows them to offer higher data caps or even alleviate the
need to introduce them. From an ISP perspective, cooperation with a CDI offers
the possibility to do global traffic and peering management through an improved
awareness of traffic across the whole network. For example, peering agreements
with CDIs can offer cooperation in exchange for reduced costs to CDIs. This can
be an incentive for CDIs to peer with ISPs, and an additional revenue for an
ISP, as such reduced prices can attract additional peering customers.
Furthermore, collaboration with CDIs has the potential to reduce the
significant overhead due to the handling of customer complaints that often do
not stem from the operation of the ISP but the operation of CDIs~\cite
{ControllingDataCloud}. Through this, ISPs can identify and mitigate congestion
in content delivery, and react to short disturbances caused by an increased
demand of content from CDIs by communicating these incidents back directly to
the source.

\subsection{Effect on End-users}\label{sec:Users-Incentive} 

Collaboration between ISPs and CDIs in content delivery empowers end-users to
obtain the best possible quality of experience. As such, this creates an
incentive for end-users to support the adoption of collaboration by both ISPs
and CDIs. For example, an ISP can offer more attractive products, \ie higher
bandwidth or lower prices, since it is able to better manage the traffic inside
its network.  Also, thanks to better traffic engineering, ISPs can increase
data caps on their broadband offers, making the ISP more attractive to
end-users. Moreover, CDIs that are willing to jointly deliver content can offer
better quality of experience to end-users. This can even be done through
premium services offered by the CDI to its customers. For example, CDIs
delivering streaming services can offer higher quality videos to end-users
thanks to better server assignment and network engineering.
