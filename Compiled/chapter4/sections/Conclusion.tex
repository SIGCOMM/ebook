\section{Conclusion}\label{sec:conclusion}

People value the Internet for the content and the applications it makes
available~\cite{CCN}. For example, the demand for online entertainment and web
browsing has exceeded 70\% of the peak downstream traffic in the United
States~\cite{sandvine}. Recent traffic
studies~\cite{TrafficTypesGrowth:2011,arbor,PADIS2010} show that a large
fraction of Internet traffic is originated by a small number of Content
Distribution Infrastructures (CDIs). Major CDIs include highly popular
rich-media sites such as YouTube and Netflix, One-Click Hosters (OCHs), \eg
RapidShare, Content Delivery Networks (CDNs) such as Akamai, and hyper-giants,
\eg Google and Yahoo!. Gerber and Doverspike~\cite{TrafficTypesGrowth:2011}
report that a few CDIs account for more than half of the traffic in a US-based
Tier-1 carrier.

To cope with the increasing demand for content, CDIs deploy massively
distributed infrastructures~\cite{ImprovingPerformanceInternet2009} to
replicate content and make it accessible from different locations in the
Internet~\cite{CDNsec2009,Cartography}.  Not all CDIs are built upon the same
philosophy, design, or technology. For example, a CDI can be operated
independently by deploying caches in different networks, by renting space in
datacenters, or by building datacenters. Furthermore, some CDIs are operated by
ISPs, some by Content Producers, or in the case of Peer-to-Peer networks, by
self-organized end users. Accordingly, we give an overview of the spectrum of
CDI solutions.

CDIs often struggle in mapping users to the best server, regardless of whether
the best server is the closest, the one with the most capacity, or the one
providing the lowest delay.  The inability of CDIs to map end users to the
right server stems from the fact that CDIs have limited visibility into ISP
networks, \ie a CDI has incomplete knowledge of an ISP's set up, operation, and
current state.  Thus, in this book chapter, we propose viewing the challenges
that CDIs and ISPs face as an opportunity: \emph{to collaborate}.  We point out
the opportunities and incentives for all parties---CDIs, ISPs and end
users---to get involved. This collaboration may ultimately lead to major
changes in the way that content is distributed across the Internet.

Accordingly, we review the proposed enablers and building blocks for
collaboration ranging from the P2P oracle service, P4P, Ono, and PaDIS, to the
IETF activities~\cite{afs-cispp2pcip-ccr07,p4p,taming,PADIS2010,ALTO-WG,CDNi}. To
illustrate the benefits of collaboration between applications and networks, we
provide two use-cases: P2P and traffic engineering. The main take away is that
substantial benefits for all involved parties are obtainable.

Upcoming trends include virtualization and the Cloud. These trends offer new
ways of collaborative deployment of content delivery infrastructure if combined
with the proposed enablers for collaboration.  Accordingly, we propose Network
Platform as a Service (\Netpaas), which allows CDIs and ISPs to cooperate not
only on user assignment, but on dynamically deploying and removing servers and
thus scaling content delivery infrastructure on demand.

We believe that ISP-CDN collaboration and \Netpaas can play a significant role
in the future content delivery ecosystem.  Most of the collaboration enablers,
however, have not yet been deployed in the wild, and therefore only the future
will tell if the Internet takes advantage of these opportunities.
