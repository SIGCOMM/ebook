% This work is licensed under
% http://creativecommon.org/licenses/by/3.0/
\section{Introduction}
\label{sec:sec1}

The Internet is increasingly mobile.  
Users access Internet services from mobile devices that move from one
wireless access point to another, or switch between WiFi and cellular 
network connectivity.  
Ubiquitous computing relies on sensors and actuators attached to vehicles,
portable objects, and animals as well as people.
Applications 
provide customers with application-level identities that can be
used to reach them at whichever device they are currently using.
Increasingly, software runs on virtual machines that can migrate from one
physical server, or even one data center, to another.

We define network
mobility as the capability that allows a communicating entity
to continue to communicate over a network, despite the fact that its
location at (or binding to) a lower-level communicating entity is changing.
Further, we focus on so-called ``seamless'' mobility, in which the
high-level entity's communication channels are preserved throughout
the change.

It is important to note that this definition applies at all conceptual
levels.
A person can have an identifier as a communicating entity, and can be
mobile by moving from one networked device to another.
A device such as a
cellphone can have an identifier, and can be mobile by moving from one
network attachment point to another.
An interface on a device, such as an Ethernet interface, can have an
identifier and be mobile by moving within or between local area networks.

In recent years, many mechanisms have emerged for supporting 
mobility.
A recent survey~\cite{rfc6301} cites 22 Internet protocols dating from 1991
to 2009, including most prominently Mobile IPv4, Mobile IPv6,
MSM-IP, HIP, MOBIKE,
Cellular IP, HAWAII,
ILNP, and LISP Mobile Node.
In other contexts mobility is supported by:
\begin{itemize}
\item
Ethernet LANs and VLANs, which allow an interface to retain its IP address 
as the host moves within the LAN;

\item 
the scalable ``flat'' routing architectures 
SEATTLE \cite{seattle}, PortLand~\cite{portland},
VL2~\cite{vl2}, NVP~\cite{nvp}, and
Rbridges/TRILL~\cite{rbridges,rfc5556}
that also naturally support routing to an interface that
retains its addresses as the host move;

\item
injecting the IP address of a mobile machine into existing routing protocols
such as OSPF and BGP~\cite{connexion,smog};

\item
the General Packet Radio Service (GPRS) Tunneling Protocol (GTP),
which supports mobility in most cellular networks;

\item
other research proposals including
TCP Migrate \cite{tcp-migrate}, Serval \cite{serval}, 
and the Internet Indirection Infrastructure~\cite{i3};

\item
application-level protocols such as 
the Session Initiation Protocol (SIP)~\cite{mobilesip}.
\end{itemize}

\noindent
Each well-known proposal tends to spawn a family of variants, so
the total number is probably in the hundreds and growing.

These various mechanisms operate at different levels, and make
different assumptions about naming, routing, session
protocols, scale, security, and the cooperation of
remote endpoints or multiple administrative domains.  
Because the
community lacks a common framework for describing and comparing
mobility mechanisms, their relationships are poorly understood.
Comparisons tend to be based on superficial characteristics rather
than inherent ones.  
Quantitative comparison must be based on
labor-intensive prototyping and measurement or simulation.

This book chapter has two goals:
\begin{itemize}
\item
To describe and compare existing proposals for implementing mobility.
\item
To map out a design space in which
new mobility mechanisms can be discovered,
evaluated, and exploited.
\end{itemize}

\noindent
To achieve these goals,
we explain mobility in a new way.
We begin by defining a common framework for describing network
architectures.
This framework is
called the {\it geomorphic view} of networking, and is
introduced in
Section~\ref{sec:sec2}.

The geomorphic view has been developed to be simple, modular,
comprehensive, and formalizable.
It supports the first goal by providing a precise, unique description
of each mobility mechanism
that omits inessential detail while exposing subtle differences
and important engineering trade-offs.

The common framework supports the second goal in several ways.
It allows us to generalize over the implementations of mobility,
showing that they are all instances of two major patterns.
It also allows us to understand how 
different instances of the mobility patterns at different places in a
network
architecture can be composed, generating a potentially large design
space to be explored.

Section~\ref{sec:sec3} of this chapter introduces the two major
patterns---\emph{dynamic-routing mobility}
and \emph{session-location mobility}---for implementing mobility.
If they are both implemented within an IP layer, the difference centers
on whether a mobile machine
retains its IP address when
it moves, or changes its IP address and updates its correspondents.
Each pattern has a completely different set of subsidiary design
decisions and resource costs.

The next sections 
use the patterns to describe and compare many of the most
important proposals for mobility.  
Sections~\ref{sec:sec4} and
\ref{sec:sec5} compare different ways to implement
dynamic-routing mobility,
with a non-hierarchical name space ({\it e.g.,}
MAC addresses in a local area
network) or a hierarchical name
space ({\it e.g.,} IP addresses in the wide
area), respectively.  
Section~\ref{sec:sec6} compares four prominent protocols for
session-location mobility at different stages in the IETF
standardization process.

Section~\ref{sec:sec7} of the chapter turns to
a more systematic exploration of the
design space, first showing that there may be some freedom concerning
where in an architecture a particular kind of mobility is handled.
The section also discusses composition of mobility
mechanisms.
As an example, we illustrate how Mobile IPv6 is a
composition of both the dynamic-routing mobility and session-location
mobility design patterns.

Finally, Section~\ref{sec:sec8} surveys several topics closely
related to mobility, including multihoming, anycast services, site
mobility, incremental deployment of mobility protocols, and security
issues for mobility.  
The chapter ends with a brief
conclusion outlining several more advanced areas of study.
